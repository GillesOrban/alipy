\section{Introduction}
\label{introduction}

\subsection{ASCII tables in astronomy and science}
\label{atables_astronomy}
ASCII tables are one of the major data exchange formats used in science.
In astronomy, which is the background of \AAD's developers, ASCII tables are
used for a variety of things like object lists, line lists or
even spectra. Every person working in astronomy has to deal with
ASCII data, and there are various ways of doing so. Some use the {\tt awk}
scripting language, some transfer the ASCII tables to FITS tables and then
work on the FITS data, some use IDL routines. Most of those approaches need
individual efforts (such as preparing a format file for the
transformation to FITS) whenever there is a new kind of ASCII table
with e.g. a different number of columns.\\


\subsection{The project goal}
\label{project_goal}
Within the \AAD project we envision a module which can be
used to work on {\bf all} kinds of ASCII tables. The module provides a
convenient tool such that the user easily can:

\begin{itemize}
\item read in ASCII tables;
\item manipulate table elements;
\item save the modified ASCII table;
\item read and write meta data such as column names and units;
\item combine several tables;
\item delete/add rows and columns;
\item manage metadata in the table headers.
\end{itemize}

\subsection{Why python?}
\label{why_python}
Python \index{python}(\htmladdnormallink{www.python.org}{www.python.org}) is
in the process of becomming the programming
language of choice for astronomers and scientists in general, both for
interactive data analysis as well as for large scale software development.
A number of interfaces such as PyRAF\index{PyRAF}
(\htmladdnormallink{http://www.stsci.edu/resources/software\_hardware/pyraf}{http://www.stsci.edu/resources/software_hardware/pyraf})
or PyFITS \index{PyFITS}\newline
(\htmladdnormallink{http://www.stsci.edu/resources/software\_hardware/pyfits}{http://www.stsci.edu/resources/software_hardware/pyfits})
have already been written to bridge the gap between widely used
astronomical software
packages, data formats and Python.

This makes the  development of the \AAD module for Python a natural
choice.
Within Python, the \AAD module may be used interactively, within small scripts,
in data reduction tasks and even in data bases.

\subsection{Design considerations}
\label{design_considerations}
In general, the ASCII tables used in astronomy have a relatively small size.
As an example, the size of the Wide Field Camera catalogue of
Hubble Ultra Deep Field
(\htmladdnormallink{http://www.stsci.edu/hst/udf}{http://www.stsci.edu/hst/udf})
is only 2.2\,MB. Handling those amounts of data is not a time consuming task for modern day computers. As a consequence, computational speed is not a prime issue in software design and
construction, and there was no attempt to implement \AAD as a particularly fast
module. The focus was rather to maximizing convenience and ensuring a
steep learning curve for the users.\\

\subsection{The SExtractor table format}
\label{sex_format}
There are many ways to store meta data such column name and units in a file
together with the table data. Instead of defining our own, proprietary format
within the \AAD module, we have chosen to support the
SExtractor\index{SExtractor} header scheme.

This means that the module can read ASCII tables which follow the SExtractor
format and extract all column information from the file
(see Sect.\ \ref{SExtractor_data}). The module also offers to write this information
in the SExtractor format back into file.

In the SExtractor format the meta data is stored at the beginning of the
file:
\begin{small}
\begin{verbatim}
#   1 NUMBER          Running object number
#   2 XWIN_IMAGE      Windowed position estimate along x              [pixel]
#   3 YWIN_IMAGE      Windowed position estimate along y              [pixel]
#   4 ERRY2WIN_IMAGE  Variance of windowed pos along y                [pixel**2]
#   5 AWIN_IMAGE      Windowed profile RMS along major axis           [pixel]
#   6 ERRAWIN_IMAGE   RMS windowed pos error along major axis         [pixel]
#   7 BWIN_IMAGE      Windowed profile RMS along minor axis           [pixel]
#   8 ERRBWIN_IMAGE   RMS windowed pos error along minor axis         [pixel]
#   9 MAG_AUTO        Kron-like elliptical aperture magnitude         [mag]
#  10 MAGERR_AUTO     RMS error for AUTO magnitude                    [mag]
#  11 CLASS_STAR      S/G classifier output
#  12 FLAGS           Extraction flags
  1 100.523  11.911   2.783 0.0693 2.078 0.0688  -5.3246   0.0416  0.00  19
  2 100.660   4.872   7.005 0.1261 3.742 0.0989  -6.4538   0.0214  0.00  27
  3 131.046  10.382   1.965 0.0681 1.714 0.0663  -4.6836   0.0524  0.00  17
  4 338.959   4.966  11.439 0.1704 4.337 0.1450  -7.1747   0.0173  0.00  25
\end{verbatim}
\end{small}

The format is rather simple, but nevertheless offers the possibility to save the
essential column information.

Potential users who would need or prefer other formats can:
\begin{itemize}
\item try to convince us that the alternative format is worth the implementation;
\item sub-class the relevant module classes and implement the format support by
themselves. We would certainly offer help for this.
\end{itemize}

\subsection{The usage of the User Manual}
\label{manualusage}
With more than 70 pages this User Manual is quite voluminous. But to start
reading at page 1 and working through it all is pointless. Our
recommendations on how to proceed are the following:
\begin{itemize}
  \item install the module with the the help of Sect.\ \ref{installation};
  \item get a first impression on \AAD browsing through Sect.\
  \ref{allinonechapter};
  \item start working with the module;
  \item use Sect.\ \ref{details} with its
  detailed examples as a reference manual\index{reference manual}, preferably in
  the {\tt html} form (available at the\index{project site}
  project site
  \htmladdnormallink{http://www.stecf.org/software/PYTHONtools/astroasciidata/}{http://www.stecf.org/software/PYTHONtools/astroasciidata/}), when looking for a certain functionality or features you need.
\end{itemize}
\subsection{Feedback}
\label{feedback}
Feedback in any form, suggestions,
critics, comments, development requests, is very much welcome and will
certainly contribute to improve the next versions of the module.
The feedback should be sent directly to the developers or to\newline
{\tt AstroAsciiData@stecf.org}.
