%
% Section the AsciiColumn class
%
\subsection{The AsciiColumn class}
\label{acclass}
\index{classes}\index{classes!AsciiColumn}
\index{AsciiColumn class}
The \ac class is the the second important class in the \AAD module.
The \ac manages all column related issues, which means that even the
actual data is stored in \ac objects. These \ac object are accessed
via the \ad object, either specifying the column name
(such as e.g.\ \verb+adata_object['diff1']+) or the column index
(such as e.g.\ \verb+adata_object[3]+).

\subsubsection{AsciiColumn data}
\label{acd}
\index{class data}\index{AsciiColumn!data}
\ac objects contain some information which is important to the user
and can be used in the processing. Although it is possible,
this class data should {\bf never} be changed directly by the user.
All book-keeping is done internally.

\prgrf{Data}
\begin{tabular}{lcl}
colname &{\it string}& file name associated to the object\\
\end{tabular}


\subsubsection{AsciiColumn method get}
\label{acm_get}
\index{methods}\index{methods!AsciiColumn!get}
\index{get}
%
This method retrieves one list element of an \ac instance.
The element is specified with the row number.

\prgrf{Usage}
elem = acol\_object[row]\\
{\it or}\\
elem = operator.getitem(acol\_object, row)

\prgrf{Parameters}
\begin{tabular}{lcl}
row &{\it int}& the row number of the entry to be replaced\\
\end{tabular}


\prgrf{Return}
- the requested column element

\prgrf{Examples}
\begin{enumerate}
\item Retrieve and print the first element of the \ac instance which is the
third column of the \ad instance {\tt 'exa'}:
\begin{small}
\begin{verbatim}
>>> exa = asciidata.open('some_objects.cat')
>>> print exa
#
# most important objects
#
    1  1.0  red  23.08932 -19.34509
    2  9.5 blue  23.59312 -19.94546
    3  3.5 blue  23.19843 -19.23571
>>> elem = exa[2][0]
>>> print elem
 red
>>>
\end{verbatim}
\end{small}
\end{enumerate}

\subsubsection{AsciiData method set}
\label{acm_set}
\index{methods}\index{methods!AsciiColumn!set}
\index{set}
%
This methods sets list members, which means elements, of an \ac instance.
The list member to be changed is addressed via their row number.

\prgrf{Usage}
acol\_object[row] = an\_entry\\
{\it or}\\
operator.setitem(acol\_object, row, adata\_column)

\prgrf{Parameters}
\begin{tabular}{lcl}
row &{\it int}& the row number of the entry to be replaced\\
an\_entry &{\it string/integer/float}& the data to replace the previous entry\\
\end{tabular}
%

\prgrf{Return}
-

\prgrf{Examples}
\begin{enumerate}
\item Replace the third entry of the column which is the third column
in the \ad instance {\tt 'exa'}:
\begin{small}
\begin{verbatim}
>>> print exa
#
# most important objects
#
    1  1.0  red  23.08932 -19.34509
    2  9.5 blue  23.59312 -19.94546
    3  3.5 blue  23.19843 -19.23571
>>> exa[2][2] = 'green'
>>> print exa
#
# most important objects
#
    1  1.0  red  23.08932 -19.34509
    2  9.5 blue  23.59312 -19.94546
    3  3.5 green  23.19843 -19.23571
>>>
\end{verbatim}
\end{small}
\end{enumerate}


\subsubsection{AsciiColumn method len()}
\label{acm_len}
\index{methods}\index{methods!AsciiColumn!len()}
\index{len()}
%
This method defines a length of an \ac instance, which equals the
number of row in the \ac.

\prgrf{Usage}
len(ac\_object)


\prgrf{Parameters}
-

\prgrf{Return}
- the length (= number of rows) of the \ac

\prgrf{Examples}
\begin{enumerate}
\item Print the length of the fifth column onto the screen:
\begin{small}
\begin{verbatim}
>>> exa = asciidata.open('some_objects.cat')
>>> print exa
#
# most important objects
#
    1  1.0  red  23.08932 -19.34509
    2  9.5 blue  23.59312 -19.94546
    3  3.5 blue  23.19843 -19.23571
>>> print len(exa[4])
3
>>>
\end{verbatim}
\end{small}
\end{enumerate}


\subsubsection{AsciiColumn iterator type}
\label{acm_iterator}
\index{methods}\index{methods!AsciiColumn!iterator}
\index{iterator}
%
This defines an iterator over an \ac instance. The iteration is finished after
{\tt acolumn\_object.nrows} calls and returns each element in subsequent calls.
Please note that it is {\bf not} possible to change these elements.

\prgrf{Usage}
for element in acolumn\_object:\\
... $<do\ something>$

\prgrf{Parameters}
-

\prgrf{Return}
-

\prgrf{Examples}
\begin{enumerate}
\item Iterate over an \ac instance and print the elements:
\begin{small}
\begin{verbatim}
>>> print exa
#
# most important objects
#
    1  1.0  red  23.08932 -19.34509
    2  9.5 blue  23.59312 -19.94546
    3  3.5 blue  23.19843 -19.23571
>>> acol = exa[4]
>>> for element in acol:
...     print element
...
-19.34509
-19.94546
-19.23571
>>>
\end{verbatim}
\end{small}
\end{enumerate}



\subsubsection{AsciiColumn method copy()}
\label{acm_copy}
\index{methods}\index{methods!AsciiColumn!copy()}
\index{copy()}
This method generates a so-called {\it deep copy} \index{deep copy}
of a column. This means the copy is not only a reference to an
existing column, but a real copy with all data.

\prgrf{Usage}
adata\_object[colname].copy()


\prgrf{Parameters}
-

\prgrf{Return}
- the copy of the column

\prgrf{Examples}
\begin{enumerate}
\item Copy the column 5 of \ad object 'example2' to column 2 of \ad
object 'example1'
\begin{small}
\begin{verbatim}
>>> print example1
#
# Some objects in the GOODS field
#
unknown  189.2207323  62.2357983  26.87  0.32
 galaxy            *  62.2376331  24.97  0.15
   star  189.1409453  62.1696844  25.30  0.12
 galaxy  188.9014716  62.2037839  25.95  0.20
>>> print example2
@
@ Some objects in the GOODS field
@
unknown   $  189.2207323 $  62.2357983 $  26.87 $  0.32
 galaxy   $            * $  62.2376331 $  24.97 $  0.15
   star   $  189.1409453 $  62.1696844 $  25.30 $     *
        * $  188.9014716 $           * $  25.95 $  0.20
>>> example1[1] = example2[4].copy()
>>> print example1
#
# Some objects in the GOODS field
#
unknown  0.32  62.2357983  26.87  0.32
 galaxy  0.15  62.2376331  24.97  0.15
   star     *  62.1696844  25.30  0.12
 galaxy  0.20  62.2037839  25.95  0.20
\end{verbatim}
\end{small}
\end{enumerate}

\subsubsection{AsciiColumn method get\_format()}
\label{acm_getformat}
\index{methods}\index{methods!AsciiColumn!get\_format()}
\index{get\_format()}
The method returns the format of the \ac object
The format description in \AAD is taken from Python. The Python Library
Reference
(\htmladdnormallink{Chapt.\ 2.3.6.2 in Python 2.5
}{http://www.python.org/doc/2.4.2/lib/typesseq-strings.html\#l2h-211}) gives a list of all possible formats.

\prgrf{Usage}
adata\_object[colname].get\_format()

\prgrf{Parameters}
-

\prgrf{Return}
- the format of the \ac object

\prgrf{Examples}
\begin{enumerate}
\item Get the format of \ac 0:
\begin{small}
\begin{verbatim}
>>> print example2
@
@ Some objects in the GOODS field
@
unknown   $  189.2207323 $  62.2357983 $  26.87 $  0.32
 galaxy   $            * $  62.2376331 $  24.97 $  0.15
   star   $  189.1409453 $  62.1696844 $  25.30 $     *
        * $  188.9014716 $           * $  25.95 $  0.20
>>> example2[0].get_format()
'% 9s'
\end{verbatim}
\end{small}
\end{enumerate}

\subsubsection{AsciiColumn method get\_type()}
\label{acm_gettype}
\index{methods}\index{methods!AsciiColumn!get\_type()}
\index{get\_type()}
The method returns the type of an \ac object

\prgrf{Usage}
adata\_object[colname].get\_type()

\prgrf{Parameters}
-

\prgrf{Return}
- the type of the \ac

\prgrf{Examples}
\begin{enumerate}
\item Get the type of \ac 0:
\begin{small}
\begin{verbatim}
>>> print example2
@
@ Some objects in the GOODS field
@
unknown   $  189.2207323 $  62.2357983 $  26.87 $  0.32
 galaxy   $            * $  62.2376331 $  24.97 $  0.15
   star   $  189.1409453 $  62.1696844 $  25.30 $     *
        * $  188.9014716 $           * $  25.95 $  0.20
>>> example2[0].get_type()
<type 'str'>
\end{verbatim}
\end{small}
\end{enumerate}


\subsubsection{AsciiColumn method get\_nrows()}
\label{acm_getnrows}
\index{methods}\index{methods!AsciiColumn!get\_nrows()}
\index{get\_nrows()}
This method offers a way to derive the number of rows in a \ac instance.

\prgrf{Usage}
acolumn\_object.get\_nrows()

\prgrf{Parameters}
-

\prgrf{Return}
- the number of rows

\prgrf{Examples}
\begin{enumerate}
\item get the number of rows in the column named 'column1':
\begin{small}
\begin{verbatim}
>>> exa = asciidata.open('sort_objects.cat')
>>> exa['column1'].get_nrows()
12
>>> print exa
    1     0     1     1
    2     1     0     3
    3     1     2     4
    4     0     0     2
    5     1     2     1
    6     0     0     3
    7     0     2     4
    8     1     1     2
    9     0     1     5
   10     1     2     6
   11     0     0     6
   12     1     1     5
>>>
\end{verbatim}
\end{small}
\end{enumerate}

\subsubsection{AsciiColumn method get\_unit()}
\label{acm_getunit}
\index{methods}\index{methods!AsciiColumn!get\_unit()}
\index{get\_unit()}
%
The method returns the unit of an \ac instance. If there is not unit defined,
a string with zero length is returned (\lq\rq)

\prgrf{Usage}
acolumn\_object.get\_unit()

\prgrf{Parameters}
-

\prgrf{Return}
- the unit of the column

\prgrf{Examples}
\begin{enumerate}
\item Print the overview of the \ac with index 1:
\begin{small}
\begin{verbatim}
test>more some_objects.cat
# 1 NUMBER          Running object number
# 2 X_Y
# 3 COLOUR
# 4 RA              Barycenter position along world x axis          [deg]
# 5 DEC             Barycenter position along world y axis          [deg]
#
# most important objects
#
1 1.0  red 23.08932 -19.34509
2 9.5 blue 23.59312 -19.94546
3 3.5 blue 23.19843 -19.23571
test>python
Python 2.4.2 (#1, Nov 10 2005, 11:34:38)
[GCC 3.3.3 20040412 (Red Hat Linux 3.3.3-7)] on linux2
Type "help", "copyright", "credits" or "license" for more information.
>>> import asciidata
>>> exa = asciidata.open('some_objects.cat')
>>> print exa['RA'].get_unit()
deg
>>>
\end{verbatim}
\end{small}
\end{enumerate}


\subsubsection{AsciiColumn method info()}
\label{acm_info}
\index{methods}\index{methods!AsciiColumn!info()}
\index{info()}
The method gives an overview on an \ac object including its type,
format and the number of elements.

Its focus is on interactive work.
All information can also be retrieved by other methods in a machine
readable format.

\prgrf{Usage}
adata\_object[colname].info()

\prgrf{Parameters}
-

\prgrf{Return}
- the overview on the \ac object

\prgrf{Examples}
\begin{enumerate}
\item Print the overview of the \ac with index 1:
\begin{small}
\begin{verbatim}
>>> print example2
@
@ Some objects in the GOODS field
@
unknown   $  189.2207323 $  62.2357983 $  26.87 $  0.32
 galaxy   $            * $  62.2376331 $  24.97 $  0.15
   star   $  189.1409453 $  62.1696844 $  25.30 $     *
        * $  188.9014716 $           * $  25.95 $  0.20
>>> print example2[1].info()
Column name:        column2
Column type:        <type 'float'>
Column format:      ['% 11.7f', '%12s']
Column null value : ['*']
\end{verbatim}
\end{small}
\end{enumerate}

\subsubsection{AsciiColumn method reformat()}
\label{acm_reformat}
\index{methods}\index{methods!AsciiColumn!reformat()}
\index{reformat()}
The method gives a new format\index{format} to an \ac object. Please note
that the new format does {\bf not} change the column content,
but only the string representation of the content.
The format description in \AAD is taken from Python. The Python Library
Reference
(\htmladdnormallink{Chapt.\ 2.3.6.2 in Python
2.5}{http://www.python.org/doc/2..2/lib/typesseq-strings.html\#l2h-211}) gives a list of all possible formats.


\prgrf{Usage}
adata\_object[colname].reformat('newformat')

\prgrf{Parameters}
\begin{tabular}{lcl}
new\_format &{\it string}& the new format of the \ac \\
\end{tabular}

\prgrf{Return}
-

\prgrf{Examples}
\begin{enumerate}
\item Change the format of the \ac with index 1:
\begin{small}
\begin{verbatim}
>>> print example2
@
@ Some objects in the GOODS field
@
unknown   $  189.2207323 $  62.2357983 $  26.87 $  0.32
 galaxy   $            * $  62.2376331 $  24.97 $  0.15
   star   $  189.1409453 $  62.1696844 $  25.30 $     *
        * $  188.9014716 $           * $  25.95 $  0.20
>>> example2[1].reformat('% 6.2f')
>>> print example2
@
@ Some objects in the GOODS field
@
unknown   $  189.22 $  62.2357983 $  26.87 $  0.32
 galaxy   $       * $  62.2376331 $  24.97 $  0.15
   star   $  189.14 $  62.1696844 $  25.30 $     *
        * $  188.90 $           * $  25.95 $  0.20
\end{verbatim}
\end{small}
\end{enumerate}

\subsubsection{AsciiColumn method rename()}
\label{acm_rename}
\index{methods}\index{methods!AsciiColumn!rename()}
\index{rename()}
The method changes the name on \ac object.

\prgrf{Usage}
adata\_object[colname].rename('newname')

\prgrf{Parameters}
\begin{tabular}{lcl}
newname &{\it string}& the filename to save the \ad object to\\
\end{tabular}

\prgrf{Return}
-

\prgrf{Examples}
\begin{enumerate}
\item Change the column name from 'column1' to 'newname':
\begin{small}
\begin{verbatim}
>>> print example2[3].info()
Column name:        column4
Column type:        <type 'float'>
Column format:      ['% 5.2f', '%6s']
Column null value : ['*']

>>> example2[3].rename('newname')
>>> print example2[3].info()
Column name:        newname
Column type:        <type 'float'>
Column format:      ['% 5.2f', '%6s']
Column null value : ['*']
\end{verbatim}
\end{small}
\end{enumerate}

\subsubsection{AsciiColumn method tonumarray()}
\label{acm_tonumarray}
\index{methods}\index{methods!AsciiColumn!tonumarray()}
\index{tonumarray()}
The method converts the content of an \ad object into a numarray
object. Note that this is only possible if there are no Null-entries
in the column, since numarray would not allow these Null-entries.

\prgrf{Usage}
adata\_object[colname].tonumarray()

\prgrf{Parameters}
-

\prgrf{Return}
- the \ac content in a numarray object.

\prgrf{Examples}
\begin{enumerate}
\item Convert the column with the index 3 to a numarray object:
\begin{small}
\begin{verbatim}
>>> print example2
@
@ Some objects in the GOODS field
@
unknown   $  189.2207323 $  62.2357983 $  26.87 $  0.32
 galaxy   $            * $  62.2376331 $  24.97 $  0.15
   star   $  189.1409453 $  62.1696844 $  25.30 $     *
        * $  188.9014716 $           * $  25.95 $  0.20
>>> numarr =  example2[3].tonumarray()
>>> numarr
array([ 26.87,  24.97,  25.3 ,  25.95])
\end{verbatim}
\end{small}
\end{enumerate}

\subsubsection{AsciiColumn method tonumpy()}
\label{acm_tonumpy}
\index{methods}\index{methods!AsciiColumn!tonumpy()}
\index{tonumpy()}
The method converts the content of an \ad object into a numpy
object. Columns without Null-entries are converted to numpy array
objects, columns with Null-entries become a numpy masked arrays
(see numpy manual for details).

\prgrf{Usage}
adata\_object[colname].tonumpy()

\prgrf{Parameters}
-

\prgrf{Return}
- the \ac content in a numpy (-masked) object.

\prgrf{Examples}
\begin{enumerate}
\item Convert the column with index 3 to a numpy object:
\begin{small}
\begin{verbatim}
>>> print example
@
@ Some objects in the GOODS field
@
unknown   $ 189.2207323$ 62.2357983$ 26.87$ 0.32
 galaxy   $        Null$ 62.2376331$ 24.97$ 0.15
   star   $ 189.1409453$ 62.1696844$ 25.30$ Null
      Null$ 188.9014716$       Null$ 25.95$ 0.20
>>> nump = example[3].tonumpy()
>>> print nump
[ 26.87  24.97  25.3   25.95]
\end{verbatim}
\end{small}
\item Convert the column with index 2 to a numpy object. This column contains
a Null-entry, thus it is converted to a masked array:
\begin{small}
\begin{verbatim}
>>> print example
@
@ Some objects in the GOODS field
@
unknown   $ 189.2207323$ 62.2357983$ 26.87$ 0.32
 galaxy   $        Null$ 62.2376331$ 24.97$ 0.15
   star   $ 189.1409453$ 62.1696844$ 25.30$ Null
      Null$ 188.9014716$       Null$ 25.95$ 0.20
>>> nump = example[2].tonumpy()
>>> type(nump)
<class 'numpy.core.ma.MaskedArray'>
>>> nump[3]
array(data =
 999999,
      mask =
 True,
      fill_value=999999)
\end{verbatim}
\end{small}
\end{enumerate}


\subsubsection{AsciiColumn method set\_unit()}
\label{acm_set_unit}
\index{methods}\index{methods!AsciiColumn!set\_unit()}
\index{set\_unit()}
The method sets the unit for a given column. Already existing
units are just replaced.

\prgrf{Usage}
adata\_object[colname].set\_unit(acol\_unit)

\prgrf{Parameters}
\begin{tabular}{lcl}
acol\_unit &{\it string}& the new column unit\\
\end{tabular}

\prgrf{Return}
-

\prgrf{Examples}
\begin{enumerate}
\item Set a unit for the column {\sl FLAGS}:
\begin{small}
\begin{verbatim}
>>> print sm
# 1  NUMBER  Running object number
# 2  X_IMAGE  Object position along x  [pixel]
# 3  Y_IMAGE  Object position along y  [pixel]
# 4  FLAGS  Extraction flags
    2  379.148  196.397     3
    3  177.377  199.843     4
    1  367.213  123.803     8
>>> sm['FLAGS'].set_unit('arbitrary')
>>> print sm
# 1  NUMBER  Running object number
# 2  X_IMAGE  Object position along x  [pixel]
# 3  Y_IMAGE  Object position along y  [pixel]
# 4  FLAGS  Extraction flags  [arbitrary]
    2  379.148  196.397     3
    3  177.377  199.843     4
    1  367.213  123.803     8
>>>
\end{verbatim}
\end{small}
\end{enumerate}

\subsubsection{AsciiColumn method set\_colcomment()}
\label{acm_set_colcomment}
\index{methods}\index{methods!AsciiColumn!set\_colcomment()}
\index{set\_colcomment()}
The method writes a comment for a column into the \ad header.

\prgrf{Usage}
adata\_object[colname].set\_colcomment(acol\_comment)

\prgrf{Parameters}
\begin{tabular}{lcl}
acol\_comment &{\it string}& the new column comment\\
\end{tabular}

\prgrf{Return}
-

\prgrf{Examples}
\begin{enumerate}
\item Set (in this case change) the column comment for the column {\sl FLAGS}:
\begin{small}
\begin{verbatim}
>>> print sm
# 1  NUMBER  Running object number
# 2  X_IMAGE  Object position along x  [pixel]
# 3  Y_IMAGE  Object position along y  [pixel]
# 4  FLAGS  Extraction flags  [arbitrary]
    2  379.148  196.397     3
    3  177.377  199.843     4
    1  367.213  123.803     8
>>> sm['FLAGS'].set_colcomment('Quality numbers')
>>> print sm
# 1  NUMBER  Running object number
# 2  X_IMAGE  Object position along x  [pixel]
# 3  Y_IMAGE  Object position along y  [pixel]
# 4  FLAGS  Quality numbers  [arbitrary]
    2  379.148  196.397     3
    3  177.377  199.843     4
    1  367.213  123.803     8
>>>
\end{verbatim}
\end{small}
\end{enumerate}

\subsubsection{AsciiColumn method get\_colcomment()}
\label{acm_get_colcomment}
\index{methods}\index{methods!AsciiColumn!get\_colcomment()}
\index{get\_colcomment()}
The method reads a column comment from an \ad column.

\prgrf{Usage}
adata\_object[colname].get\_colcomment()

\prgrf{Parameters}
-

\prgrf{Return}
- the comment string of the column

\prgrf{Examples}
\begin{enumerate}
\item Read and print the column comment of the column {\sl X\_IMAGE}:
\begin{small}
\begin{verbatim}
>>> cocomm = sm['X_IMAGE'].get_colcomment()
>>> print cocomm
Object position along x
>>>
\end{verbatim}
\end{small}
\end{enumerate}
