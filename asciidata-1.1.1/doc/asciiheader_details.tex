%
% Section the Header class
%
\subsection{The Header class}
\label{hclass}
\index{classes}\index{classes!Header}
\index{Header class}
The Header class manages the header of an \ad object.
The header contains a list of comments.
Any kind of meta-data such as column names are part of the
columns and therefore located in the AsciiColumn (see Sect.\ \ref{acclass})
class. The header object is accessed through various
methods to e.g. get or set items.

\subsubsection{Header method get}
\label{ahe_get}
\index{methods}\index{methods!Header!get}
\index{get}
%
The header class contains a method to get individual items from a header
instance via their index.

\prgrf{Usage}
header\_entry = adata\_object.header[index]\\
{\it or}\\
header\_entry = operator.getitem(adata\_object.header, index)

\prgrf{Parameters}
\begin{tabular}{lcl}
index &{\it int}& the index of the item to retrieve\\
\end{tabular}

\prgrf{Return}
- one entry of the header

\prgrf{Examples}
\begin{enumerate}
\item Retrieve the second entry of this table header:
\begin{small}
\begin{verbatim}
>>> print example
#
# most important sources!!
#
    1  1.0  red  23.08932 -19.34509
    2  9.5 blue  23.59312 -19.94546
    3  3.5 blue  23.19843 -19.23571
>>> header_entry = example.header[1]
>>> print header_entry
 most important sources!!

>>>
\end{verbatim}
\end{small}
\item Access the third entry of this table header:
\begin{small}
\begin{verbatim}
>>> print example
#
# most important sources!!
#
    1  1.0  red  23.08932 -19.34509
    2  9.5 blue  23.59312 -19.94546
    3  3.5 blue  23.19843 -19.23571
>>> example.header[2]
'\n'
>>>
\end{verbatim}
\end{small}
\end{enumerate}


\subsubsection{Header method set}
\label{ahe_set}
\index{methods}\index{methods!Header!set}
\index{set}
%
The header class contains a method to set individual items in a header.
The item is specified via its index.

\prgrf{Usage}
adata\_object.header[index] = new\_entry\\
{\it or}\\
header\_entry = operator.setitem(adata\_object.header, index, new\_entry)

\prgrf{Parameters}
\begin{tabular}{lcl}
index &{\it int}& the index of the item to be set\\
new\_entry &{\it string}& the new content of the header item\\
\end{tabular}

\prgrf{Return}
-

\prgrf{Examples}
\begin{enumerate}
\item Change the second header item:
\begin{small}
\begin{verbatim}
>>> print example
#
# most important sources!!
#
    1  1.0  red  23.08932 -19.34509
    2  9.5 blue  23.59312 -19.94546
    3  3.5 blue  23.19843 -19.23571
>>> example.header[1] = 'a new header entry?'
>>> print example
#
#a new header entry?
#
    1  1.0  red  23.08932 -19.34509
    2  9.5 blue  23.59312 -19.94546
    3  3.5 blue  23.19843 -19.23571
>>>
\end{verbatim}
\end{small}
\item Change the third header item:
\begin{small}
\begin{verbatim}
>>> print example
#
# most important sources!!
#
    1  1.0  red  23.08932 -19.34509
    2  9.5 blue  23.59312 -19.94546
    3  3.5 blue  23.19843 -19.23571
>>> example.header[2] = '   >>> dont forget leading spaces if desired!'
>>> print example
#
# most important sources!!
#   >>> dont forget leading spaces if desired!
    1  1.0  red  23.08932 -19.34509
    2  9.5 blue  23.59312 -19.94546
    3  3.5 blue  23.19843 -19.23571
>>>
\end{verbatim}
\end{small}
\end{enumerate}


\subsubsection{Header method del}
\label{ahe_del}
\index{methods}\index{methods!Header!del}
\index{del}
%
The header class contains a method to delete individual items in a header.
The item is specified via its index.

\prgrf{Usage}
del adata\_object.header[index]\\
{\it or}\\
operator.delitem(adata\_object.header, index)

\prgrf{Parameters}
\begin{tabular}{lcl}
index &{\it int}& the index of the item to be deleted\\
\end{tabular}

\prgrf{Return}
-

\prgrf{Examples}
\begin{enumerate}
\item Delete the second header item:
\begin{small}
\begin{verbatim}
>>> print example
#
# most important sources!!
#
    1  1.0  red  23.08932 -19.34509
    2  9.5 blue  23.59312 -19.94546
    3  3.5 blue  23.19843 -19.23571
>>> del example.header[1]
>>> print example
#
#
    1  1.0  red  23.08932 -19.34509
    2  9.5 blue  23.59312 -19.94546
    3  3.5 blue  23.19843 -19.23571
>>>
\end{verbatim}
\end{small}
\end{enumerate}

\subsubsection{Header method str()}
\label{ahe_str}
\index{methods}\index{methods!Header!str()}
\index{str()}
%
This method converts the entire \ah instance into a string. The {\tt print}
command called with an \ah instance as first parameter also prints the
string created using this method {\tt str()}.

\prgrf{Usage}
str(adata\_object.header)

\prgrf{Parameters}
-

\prgrf{Return}
- the string representation of the \ah instance

\prgrf{Examples}
\begin{enumerate}
\item Delete the second header item:
\begin{small}
\begin{verbatim}
>>> print example
#
# most important sources!!
#
    1  1.0  red  23.08932 -19.34509
    2  9.5 blue  23.59312 -19.94546
    3  3.5 blue  23.19843 -19.23571
>>> print example.header
#
# most important sources!!
#

>>>
\end{verbatim}
\end{small}
\end{enumerate}

\subsubsection{Header method len()}
\label{ahe_len}
\index{methods}\index{methods!Header!len()}
\index{len()}
%
The method defines the length of an \ah instance, which
equals the number of the comment entries. Please note that
empty lines are are counted as well.

\prgrf{Usage}
len(adata\_object.header)

\prgrf{Parameters}
-

\prgrf{Return}
- the length of the \ah instance

\prgrf{Examples}
\begin{enumerate}
\item Get the length of an \ah:
\begin{small}
\begin{verbatim}
>>> print example
#
# most important sources!!
#
    1  1.0  red  23.08932 -19.34509
    2  9.5 blue  23.59312 -19.94546
    3  3.5 blue  23.19843 -19.23571
>>> header_length = len(example.header)
>>> header_length
3
>>>
\end{verbatim}
\end{small}
\end{enumerate}


\subsubsection{Header iterator type}
\label{ahe_iterator}
\index{methods}\index{methods!Header!iterator}
\index{iterator}
%
This defines an iterator over an \ah instance. The iteration is finished after
{\tt len(adata\_object.header)} calls and returns each header element
in subsequent calls. Please not that it is {\bf not} possible to change these
elements.

\prgrf{Usage}
for element in adata\_object.header:\\
... $<do\ something>$

\prgrf{Parameters}
-

\prgrf{Return}
-

\prgrf{Examples}
\begin{enumerate}
\item Iterate over an \ah instance and print the elements:
\begin{small}
\begin{verbatim}
>>> print example
@
@Some objects in the GOODS field
@ -classification
@ -RA
@ -DEC
@ -MAG
@ -extent
unknown   $ 189.2207323$ 62.2357983$ 26.87$ 0.32
 galaxy   $        Null$ 62.2376331$ 24.97$ 0.15
   star   $ 189.1409453$ 62.1696844$ 25.30$ Null
      Null$ 188.9014716$       Null$ 25.95$ 0.20
>>> for h_entry in example.header:
...     print h_entry.strip()
...

Some objects in the GOODS field
-classification
-RA
-DEC
-MAG
-extent
>>>

\end{verbatim}
\end{small}
\end{enumerate}


\subsubsection{Header method reset()}
\label{ahe_res}
\index{methods}\index{methods!Header!reset()}
\index{reset()}
%
The method deletes all entries from an \ah instance and provides
a clean, empty header.

\prgrf{Usage}
adata\_object.header.reset()

\prgrf{Parameters}
-

\prgrf{Return}
-

\prgrf{Examples}
\begin{enumerate}
\item Reset an \ah instance:
\begin{small}
\begin{verbatim}
>>> print example
#
# most important sources!!
#
    1  1.0  red  23.08932 -19.34509
    2  9.5 blue  23.59312 -19.94546
    3  3.5 blue  23.19843 -19.23571
>>> example.header.reset()
>>> print example
    1  1.0  red  23.08932 -19.34509
    2  9.5 blue  23.59312 -19.94546
    3  3.5 blue  23.19843 -19.23571
>>>
\end{verbatim}
\end{small}
\end{enumerate}

\subsubsection{Header method append()}
\label{ahe_append}
\index{methods}\index{methods!Header!append()}
\index{append()}
The method appends a string or a list of strings to the header
of an \ad object.

\prgrf{Usage}
adata\_object.header.append(hlist)

\prgrf{Parameters}
\begin{tabular}{lcl}
hlist &{\it string}& the list of strings to be appended to the header\\
\end{tabular}

\prgrf{Return}
-

\prgrf{Examples}
\begin{enumerate}
\item Change the column name from 'column1' to 'newname':
\begin{small}
\begin{verbatim}
>>> print example2
@
@ Some objects in the GOODS field
@
unknown   $  189.2207323 $  62.2357983 $  26.87 $  0.32
 galaxy   $            * $  62.2376331 $  24.97 $  0.15
   star   $  189.1409453 $  62.1696844 $  25.30 $     *
        * $  188.9014716 $           * $  25.95 $  0.20
>>> example2.header.append('Now a header line is appended!')
>>> print example2
@
@ Some objects in the GOODS field
@
@ Now a header line is appended!
unknown   $  189.2207323 $  62.2357983 $  26.87 $  0.32
 galaxy   $            * $  62.2376331 $  24.97 $  0.15
   star   $  189.1409453 $  62.1696844 $  25.30 $     *
        * $  188.9014716 $           * $  25.95 $  0.20
>>> example2.header.append("""And now we try to put
... even a set of lines
... into the header!!""")
>>> print example2
@
@ Some objects in the GOODS field
@
@ Now a header line is appended!
@ And now we try to put
@ even a set of lines
@ into the header!!
unknown   $  189.2207323 $  62.2357983 $  26.87 $  0.32
 galaxy   $            * $  62.2376331 $  24.97 $  0.15
   star   $  189.1409453 $  62.1696844 $  25.30 $     *
        * $  188.9014716 $           * $  25.95 $  0.20
\end{verbatim}
\end{small}
\end{enumerate}
